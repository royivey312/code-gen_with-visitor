\documentclass {article}
\author {M.~W.~Schulte}
\title {The Format of the {\tt .o} File}
%%% Preamble written by William Shoaff
%%% Style for Notes
%%% Date January 2, 1989.

\renewcommand{\theenumi}{\arabic{enumi}}
\renewcommand{\labelenumi}{\arabic{enumi}.}
\renewcommand{\theenumii}{\alph{enumii}}
\renewcommand{\labelenumii}{(\alph{enumii})}
%\renewcommand{\theenumiii}{\roman{enumiii}}
%\renewcommand{\labelenumiii}{\roman{enumiii})}
%\renewcommand{\theenumiv}{\arabic{enumiv}}
%\renewcommand{\labelenumiv}{\arabic{enumiv})}

\newcommand{\fillin}[1]{\makebox[#1]{\hrulefill}}%%% fill in the blank command: \fillin[length]

%%% New Symbols

\newcommand{\cents}{\hbox{\rm\rlap/c}} % the cents symbol
\newcommand{\norm}[1]{\mid\mid{#1}\mid\mid} 
	% the norm symbol -- may only be used in math mode
\newcommand{\abs}[1]{\mid{#1}\mid}
	% the absolute value symbol -- may only be used in math mode
\newcommand{\integer}{\hspace*{2pt}\mbox{\tt N}\hspace*{-8pt}
		\mbox{\tt I}\hspace*{8pt}}	% the set of integers
\newcommand{\real}{\hspace*{2pt}\mbox{\tt R}\hspace*{-8pt}
		\mbox{\tt I}\hspace*{8pt}}	% the set of reals
\newcommand{\floor}{\mbox{\tt floor}} % the floor function
\newcommand{\gives}{$\rightarrow$}
\newcommand{\derives}{$\Rightarrow$}
\def\U#1{\underline{#1}}
\newcommand{\zB}{\underline{z.B.}}

%%% Page layout

\unitlength=1in
\marginparwidth=0.7in
\newcommand{\fullpage}{\oddsidemargin=0pt\textwidth=6.5in
	\topmargin=0pt\headheight=0pt\headsep=0pt
	\footskip=0pt \textheight=9in}

%%% Sectioning commands

%\mychapter{\large\bf \thechapter.~#1}{\large\bf~#1}
%\mysection{section}{\normalsize\bf}{\thesection.~}

%%% New Environments

\newtheorem{theorem}{Theorem}
\newtheorem{lemma}{Lemma}
\newtheorem{corollary}{Corollary}
\def\endProof{\hspace*{\fill}\vrule height 1em width .5em depth 0em\medskip\par}

% define for proof
\newenvironment{proof} {{\bf Proof:}}{\endProof\medskip}

% define for definition
\newenvironment{define} {\tt\medskip\noindent{\bf Definition:}\begin{flushleft}} {\end{flushleft}}

% define for problem
\newenvironment{problem} {\tt\medskip\noindent{\bf Problem:}\begin{flushleft}} {\end{flushleft}}

% define for lemma
%\newenvironment{lemma} {\tt\medskip\noindent{\bf Lemma:}\begin{flushleft}} {\end{flushleft}}

% example for examples
\newenvironment{example} {\tt\medskip\noindent{\bf Example:}\begin{flushleft}} {\end{flushleft}}

% code for pseudocode
\newenvironment{code} {\medskip\noindent\begin{verbatim}}{\end{verbatim}}

%\def\labelenumi{\Roman{enumi}.}
%\def\theenumi{\Roman{enumi}}
%\def\labelenumii{\arabic{enumii}.}
%\def\theenumii{\arabic{enumii}}
%\def\p@enumii{\theenumi.}
%\def\labelenumiii{\alph{enumiii}.}
%\def\theenumiii{\alph{enumiii}}
%\def\p@enumiii{\theenumi.\theenumii.}
%\def\labelenumiv{\arabic{enumiv}.}
%\def\theenumiv{\arabic{enumiv}}
%\def\p@enumiv{\p@enumiii.\theenumiii}
\fullpage
\begin{document}
\maketitle

The object file is an ASCII file containing  the generated code with 
intermixed directives to the linker/loader.

The instructions can be either written as approved memonics (in all upper
case) or as the equivalent numbers.  The arguements are numbers or symbols.
the forward references are handled automatically.

A number followed immediately by a colon (``:'') 
(no space between number and colon) is taken as a command 
to change the next storage location (PC)  to the number 
plus the start address of the module [specify the load address].  
The initial load address is taken to be location 0 in the code section. 

Ther are a series of linker/loader directives:\begin {description}
\item [{\tt .CODE}] indicates that the loading is to be done to the 
code section [the initial default].  
\item [{\tt .DATA}] indicates that 
the loading is to be done to the data section.
\item [{\tt .ENTRY}] followed by a number or symbol indicates the 
(relocatable) entry point.
\item [{\tt .EXTERN}] followed by a name and a parenthesized
list of numbers indicates that the name is an external symbol used at the
relative code locations in the list of value.
\item [{\tt .RELOCs}] where {\tt s} can be {\tt B}, {\tt W}, {\tt L}, or 
{\tt Q}, followed by a parenthesized list of numbers indicates
that the (relocatable) locations specified in the list 
in the current section need to be relocated.
\item [{\it name}:] (no space between name and colon) denotes a global symbol 
whose value is the (relocatable) current location in the current section.
The loader complains if you try to redefine the name, and uses the first 
occurence.
\item [{\it name}=] (no space before the equal sign)
followed by a name or a number 
denotes a global symbol whose value is value of arguement.
You can redefine the name.
\end {description}
\pagebreak

\zB,\\
SIZE = 512 .CODE 0: LPUTW HALT\\ 
A: PUSHW 0 PUSHW B PUSHW 0 PUSHW Z LGETW PUSHW SIZE MULTI \\
PUSHQ 0 GOTO\\
.ENTRY A .DATA 4: A: 32W 16: B: 25678912L

sets SIZE to the value 512, starts the CODE section (done by default also),
puts a series of instructions into the code section, declares the symbol A 
to have value 2 [PC at that point is 2], declares that the entry point is 
the label A [2 in the CODE section], switches to the DATA section, moves to 
location 4, declares a label Z with that value and puts the word 32 there, 
then moves to location 16, declares a label B and 
puts the longword 25678912 there.
\end{document}
